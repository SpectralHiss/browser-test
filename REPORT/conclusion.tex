
\chapter{Conclusion} 


What we could add to this project is adding new tests, better understanding of the resuts in a no nonsense statistics interface but we judge
that our biggest contribution is opening up the space of formalism by recognising that the quality of the data that browseraudit provides, coupled
with some logically formulated domain knowlege could lead to a practical tool that increases usability of security on the web.


As for the second part (building the data aggregator) we thought this was to be a fairly mundane programming task which has well defined behaviour
but ended up being quite tricky due to the rigidness of the data model for executions as well as strange behaviour  \
Nevertheless due to the RESTful way our server code is setup we ended up reusing a lot of the data model processing and simplifying the task
for other sides 

On the other hand it is entirely possible to find massive flaws in the different policies or the implementation of new HTML5 features in browsers in which case we will \
have the added benefit of providing insight and advancing security for all. If things go this well, then it will be a good idea to start to build channels of communication\
with browser devs. or standards people to discuss the direction of standards which might give valuable insight.\



Ultimately as a web developer it would be an added benefit and a first point of reference to build something that will make writing secure client based applications less of\
a hassle. And while precisely contains the cutting edge threats in one place gives also a manner of priorising effort compared to risks. Also giving examples for each types of\
intrustions in a digestable format would be an very beneficial.\

